\documentclass[a4paper, 10pt]{article}

\input{preamble}


\author{Вячеслав Тарасов}
\title{ Trial cross-bar controller\\
        Device guide}
\date{\today}

\begin{document}

\maketitle
\thispagestyle{empty}
\newpage

\tableofcontents
\pagestyle{myheadings}
\pagestyle{fancy}
\fancyhf{}
\setlength{\headheight}{30pt}
\renewcommand{\headrulewidth}{4pt}
\renewcommand{\footrulewidth}{2pt}
\fancyhead[R]{\leftmark}
\fancyfoot[C]{\thepage}

\newpage
\section{Рабочее окружение}

Описание всех основных директорий рабочего окружения.

\subsection{doc}

Директория с исходниками документации.
\begin{itemize}
 \item \textbf{dev\_guide.tex} - руководство разработчика в \LaTeX версии;
 \item \textbf{preamble.tex} - преамбула для \LaTeX окружения;
 \item \textbf{dev\_guide.pdf} - руководство разработчика, сохраненное в PDF формате.
\end{itemize}

\subsection{rtl}

Директория с исходниками cross-bar контроллера.
\begin{itemize}
 \item \textbf{cross\_bar\_pkg.sv} - пакет со всеми ключевыми параметрами дизайна;
 \item \textbf{cross\_bar\_top.sv} - модуль верхнего уровня контроллера;
 \item \textbf{cross\_bar\_master.sv} - модуль мультиплексоров для сигналов из slave в master;
 \item \textbf{cross\_bar\_slave.sv} - модуль мультиплексоров для сигналов из master в slave c арбитражной логикой;
 \item \textbf{cross\_bar\_arbiter.sv} - модуль арбитра.
\end{itemize}


\subsection{sim}

Содержит скрипты для запуска симуляции в Xcelium.
\begin{itemize}
 \item \textbf{src.files} - файл со списком всех исходников для симуляции.
\end{itemize}

\subsection{syn}

Директория со скриптами запуска синтеза в Genus. Используется для оценки временных/пространственных параметров блоков.
\begin{itemize}
 \item \textbf{lib} - директория с библиотеками.
\end{itemize}

Директория \textbf{cross\_bar} построена по принципу:
\begin{itemize}
 \item \textbf{syn\_genus/run\_genus} - скрипт для запуска Genus, в который передаются Tcl скрипты, описанные ниже;
 \item \textbf{syn\_genus/elab.tcl} - скрипт с командами подключения библиотек, исходников и сборки;
 \item \textbf{syn\_genus/syn.tcl} - скрипт с командами ограничений для тактового сигнала и синтеза;
 \item \textbf{syn\_genus/clean} - скрипт для очистки всех артефактов синтеза;
 \item \textbf{syn\_wrp.sv} - файл обертка для синтезируемого модуля;
\end{itemize}


\subsection{tb}

Директория с тестбенчами.
\begin{itemize}
 \item \textbf{dev} - рабочий тестбенч; используется при разработке контроллера;
 \item \textbf{rr\_arbiter} - тестбенч для тестирования арбитра;
 \item \textbf{tb\_vip\_slave.sv} - отдельный файл, в который вынесен модуль Verification IP инфраструктуры.
\end{itemize}

\newpage{}
\section{Архитектура}

Cross-bar controller позволяет осуществлять коммутацию нескольких master и нескольких slave устройств.
Параметры контроллера:
\begin{itemize}
 \item Количество master устройств (parameter \textbf{MASTER\_N}), не менее 2-х;
 \item Количество slave устройств (parameter \textbf{SLAVE\_N});
 \item Ширина шины адреса (parameter \textbf{ADDR\_W});
 \item Ширина шины данных (parameter \textbf{DATA\_W});
 \item Выбор slave устройства определяется старшими битами адреса;
 \item Арбитраж между несколькими master запросами в одно slave устройство осуществляется по дисциплине "round-robin".
\end{itemize}


\subsection{Описание портов}

\begin{table}[h!]
 \caption{Тактирование и сброс}
 \centering
  \begin{tabular}{lll}
   \hline
   Сигнал & Направление & Примечание\\
   \hline 
   clk     & вход & тактовый сигнал\\
   aresetn & вход & асинхронный сброс с активным нулем\\
   \hline 
  \end{tabular}
\end{table}

\begin{table}[h!]
 \caption{Интерфейс к master устройствам}
 \centering
  \begin{tabular}{lll}
   \hline
   Сигнал & Направление & Примечание\\
   \hline 
   master\_req                   & вход  & запрос на выполнение транзакции\\
   master\_addr[ADDR\_W - 1: 0]  & вход  & содержит адрес запроса\\
   master\_cmd                   & вход  & признак операции: 0 - read, 1 – write\\
   master\_wdata[DATA\_W - 1: 0] & вход  & содержит записываемые данные\\
   master\_ack                   & выход & сигнал-подтверждение от slave устройства\\
   master\_rdata[DATA\_W - 1: 0] & выход & содержит считываемые данные\\
   \hline 
  \end{tabular}
\end{table}

\begin{table}[h!]
 \caption{Интерфейс к slave устройствам}
 \centering
  \begin{tabular}{lll}
   \hline
   Сигнал & Направление & Примечание\\
   \hline 
   slave\_req                   & выход & запрос на выполнение транзакции\\
   slave\_addr[ADDR\_W - 1: 0]  & выход & содержит адрес запроса\\
   slave\_cmd                   & выход & признак операции: 0 - read, 1 – write\\
   slave\_wdata[DATA\_W - 1: 0] & выход & содержит записываемые данные\\
   slave\_ack                   & вход  & сигнал-подтверждение от slave устройства\\
   slave\_rdata[DATA\_W - 1: 0] & вход  & содержит считываемые данные\\
   \hline 
  \end{tabular}
\end{table}

\subsection{Тактирование и сброс}

\begin{itemize}
 \item Разделения на тактовые домены внутри нет; 
 \item Для всех триггеров внутри контроллера используется асинхронный сброс с активным нулём.
\end{itemize}

\subsection{Описание модулей}

\subsubsection{cross\_bar\_top}
Модуль верхнего уровня контроллера. Здесь происходит подключение всех основных модулей.
Все параметры задаются через пакет cross\_bar\_pkg.

\subsubsection{cross\_bar\_pkg}
Пакет со всеми глобальными для дизайна параметрами, типами и пр. Этот пакет импортируется во все модули контроллера.

\subsubsection{cross\_bar\_master}
Чистая комбинационная логика мультиплексоров сигналов ack и rdata от slave к master.

\subsubsection{cross\_bar\_slave}
Модуль содержит в себе комбинационную логику мультиплексоров сигналов req, addr, cmd и wdata от master к slave, фильтр запросов req и модуль арбитража.

\subsubsection{cross\_bar\_rr\_arbiter}
Модуль арбитра, переписанный из открытых источников, требует дополнительной проверки.
Также присутствует логика синхронизации сигналов req.

\newpage{}
\section{Функционирование}
Полностью параметризованный модуль cross-bar, позволяющий производить автоматическую коммутацию настраиваемого количества master устройств и slave устройств.

Выбор slave устройства определяется старшими битами сигналов addr.

В случае одновременного обращения нескольких master устройств к одному slave устройству осуществляется арбитраж по дисциплине ``round-robin``.



\end{document}
